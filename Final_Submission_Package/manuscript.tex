\documentclass[fleqn,10pt]{wlscirep}
\usepackage[utf8]{inputenc}
\usepackage[T1]{fontenc}
\usepackage{amsmath}
\usepackage{graphicx}
\usepackage{float}

% Point to the figures directory
\graphicspath{{./figures/}}

\title{Emergent Bistability and Persistent Activity in the Drosophila Sm Cell Connectome: A Large-Scale Simulation Study}

\author[1,*]{Author Name}
\author[1]{Supervisor Name}
\affil[1]{Department of Computational Neuroscience, Institute Name, City, Country}
\affil[*]{corresponding.author@email.com}

\begin{abstract}
The mechanisms by which small neural circuits sustain persistent activity underlie working memory and decision-making. Using the newly released FlyWire connectome, we analyze the structural and functional properties of the specific population of 4,463 Small (Sm) cells in the \textit{Drosophila} optic lobe. We combine large-scale spiking network simulations with mean-field bifurcation theory to demonstrate that this population is capable of robust bistability. By deriving an empirical gain function from Adaptive Exponential (AdEx) neuron models, we identify a critical recurrent coupling weight of $J=50$ pA that generates three fixed points in the system's nullcline analysis. We confirm these theoretical predictions with full-scale simulations ($N=4,463$), observing 100\% persistence probability of memory states. Furthermore, we characterize a "Clock Mode" regime driven by spike-frequency adaptation and quantify the network's significant modular structure ($Q \approx 0.31$, $p<0.001$). Our results bridge the gap between connectomic wiring diagrams and emergent network dynamics.
\end{abstract}

\begin{document}

\flushbottom
\maketitle

\section*{Introduction}
Recent advances in electron microscopy have yielded the first complete connectome of the adult \textit{Drosophila} brain \cite{Dorkenwald2023}. While the wiring diagram provides the structural scaffold, the functional repertoire of these circuits remains a subject of intense computational investigation. This study focuses on the Small (Sm) cell population, a group of interneurons implicated in visual processing and temporal integration.

We employ a bottom-up modeling approach, starting from individual Adaptive Exponential integrate-and-fire (AdEx) neurons \cite{Brette2005} and scaling up to the full biological connectome. We address the fundamental question: can the recurrent connectivity of Sm cells support attractor dynamics, such as persistent activity (bistability) or oscillations?

\section*{Results}

\subsection*{Structural Analysis of the Sm Connectome}
We analyzed the connectivity matrix of the complete $N=4,463$ Sm cell population retrieved from the FlyWire dataset. Spectral clustering analysis revealed a highly non-random structure. The network exhibits significant modularity ($Q = 0.3119$), which is statistically significant compared to random null models ($p < 0.001$, 50 shuffles). The eigenvalue spectrum (Fig. \ref{fig:eigen}) shows a spread consistent with a structured recurrent network rather than a random Erd\H{o}s-R\'enyi graph.

\begin{figure}[H]
\centering
\includegraphics[width=0.6\linewidth]{fig4c_eigenvalues}
\caption{\textbf{Structural properties of the Sm connectome.} Eigenvalue spectrum of the 4463x4463 connectivity matrix. The distribution and modularity score ($Q=0.31$) indicate non-random community structure.}
\label{fig:eigen}
\end{figure}

\subsection*{Mean-Field Theory and Bifurcation Analysis}
To understand the dynamic capabilities of the network, we derived a mean-field theory based on the transfer function $\Phi(I)$ of the single neuron. We measured the empirical F-I curve (Fig. \ref{fig:fi}) and established a self-consistency condition for the population release rate $r$:
\begin{equation}
r = \Phi(I_{bias} + J_{eff} \cdot r)
\end{equation}
where $J_{eff}$ is the effective recurrent coupling derived from the synaptic weight $W$ and time constant $\tau_{syn}$.
Using a synaptic weight of $W=50$ pA (corresponding to $J=50$ in our scan), our bifurcation analysis (Fig. \ref{fig:bifurcation}) reveals \textbf{three fixed points}:
1. A stable low-activity state (near 0 Hz).
2. An unstable intermediate point.
3. A stable high-activity state (saturation regime).

This S-shaped nullcline crossing the identity line confirms the theoretical capacity for bistability.

\begin{figure}[H]
\centering
\includegraphics[width=0.45\linewidth]{fig1_validation_robust}
\includegraphics[width=0.45\linewidth]{fig2_bifurcation}
\caption{\textbf{Theoretical Validation.} (Left) Empirical F-I curve of the AdEx neuron model. (Right) Bifurcation diagram showing the number of fixed points as a function of recurrent weight $J$. The green line at $J=50$ pA indicates the operating point where three fixed points (bistability) are observed.}
\label{fig:bifurcation}
\end{figure}

\subsection*{Simulation Scaling and Persistence}
We verified the mean-field predictions by simulating the full spiking network. We scaled the network size $N$ from 50 to 4,463 neurons while maintaining the effective coupling density (Fig. \ref{fig:scaling}).
The persistence probability—defined as the fraction of trials where the network maintains high activity after a transient stimulus—was found to be 1.00 (100\%) for the full-scale network ($N=4,463$). This aligned perfectly with the theoretical prediction of a stable high-activity attractor.

\begin{figure}[H]
\centering
\includegraphics[width=0.7\linewidth]{fig3_scaling_robust}
\caption{\textbf{Dynamics Scaling.} The persistence of the memory state is robust across orders of magnitude of network size. At the biological scale ($N \approx 4000$), the network reliability is maximized.}
\label{fig:scaling}
\end{figure}

\subsection*{Oscillatory "Clock" Mode}
By increasing the Spike-Frequency Adaptation parameter $b$, we observed a transition from stable fixed points to limit cycles (Fig. \ref{fig:clock}). This "Clock Mode" suggests the Sm population can switch between integrator and oscillator functions based on neuromodulatory changes to adaptation currents.

\begin{figure}[H]
\centering
\includegraphics[width=0.7\linewidth]{fig5_clock_mode}
\caption{\textbf{Clock Mode.} Raster plot showing synchronized oscillations driven by high spike-frequency adaptation.}
\label{fig:clock}
\end{figure}

\section*{Discussion}
Our results provide a bridge between the static wiring diagram of the \textit{Drosophila} optic lobe and its potential dynamics. The robust bistability found at $J=50$ pA suggests the Sm cells can serve as a discrete working memory buffer. The agreement between the mean-field bifurcation analysis and the large-scale simulation validates our reduced-order models.

\section*{Methods}

\subsection*{Neuron Model}
We used the Adaptive Exponential (AdEx) integrate-and-fire model governed by:
\begin{align}
C \frac{dV}{dt} &= -g_L(V - E_L) + g_L \Delta_T \exp\left(\frac{V - V_T}{\Delta_T}\right) - w + I_{syn} \\
\tau_w \frac{dw}{dt} &= a(V - E_L) - w
\end{align}
Parameters: $C=10$ pF, $g_L=0.5$ nS, $E_L=-60$ mV, $V_T=-45$ mV, $\tau_{syn}=5$ ms.
For the bistable regime, adaptation $b$ was set to 0. For the clock mode, $b$ was increased.

\subsection*{Connectivity Data}
Neural connectivity was obtained from the FlyWire dataset \cite{Dorkenwald2023} using the \texttt{fafbseg} API. We selected all neurons with the "Sm" prefix, resulting in a matrix of size 4,463 x 4,463.

\subsection*{Theory Scan}
We performed a parameter sweep of the recurrent weight $J$ from 0 to 80 pA. The effective coupling for the mean-field equation was calibrated to match the simulation's synaptic transfer integral.

\bibliography{references}

\end{document}
